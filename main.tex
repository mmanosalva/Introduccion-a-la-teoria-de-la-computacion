\documentclass[12pt,a4paper]{report} 
\usepackage[T1]{fontenc}
\usepackage[spanish]{babel}
\usepackage{mathrsfs}
\usepackage{amsmath}
\usepackage{amsthm}
\usepackage{amssymb}
\usepackage{yhmath}
\usepackage[breaklinks = true]{hyperref}
\usepackage{caption}
\usepackage{float}
\usepackage{graphicx}
\usepackage{tikz}
\usetikzlibrary{automata, arrows.meta, positioning}
\tikzset{
->,
>=stealth,
initial text = $ $
}
\usepackage{geometry}
 \geometry{
 a4paper,
 total={170mm,245mm},
 left=20mm,
 top=20mm,
 }


\setlength\parindent{0pt}



%%%%%%
\usepackage{newunicodechar}
\usepackage{fontspec}

\setmainfont{CMU Serif}
\setsansfont{CMU Sans Serif}
\setmonofont{CMU Typewriter Text}

\newfontface{\pencilfont}{DejaVu Sans}[Scale=MatchUppercase]

\newunicodechar{✎}{{\pencilfont ✎}}

%%%%%%%



\title{Introducción a la teoría de la computación - Solutions manual}
\author{mmanosalva\\eochoaq}
\date{\today}

\begin{document}

\maketitle
\tableofcontents
\cleardoublepage

\chapter{Sección 1.5.}
Este manual de solución de las notas de introducción a la teoría de la computación tiene como objetivo ser de ayuda para estudiantes que se encuentren viendo la materia en futuros semestres, a la fecha (2023-1) no hay ningún manual de solución de los ejercicios de este curso y nos pareció importante poder dar un poco de ayuda o soporte a las personas que en el futuro vean el curso, todo ha sido separado en secciones para que sea más sencillo encontrar los ejercicios específicos que el estudiante pueda necesitar.\\

Las soluciones que presentemos no necesariamente son correctas pero trabajaremos para que lo sean y ante cualquier error nos pueden escribir a los correos que están en la portada.\\


\textbf{Punto 1:} Primero consideramos la cadena vacía, es claro que \\

\quad $\bullet$ $\lambda^R=\lambda$\\

Esto por definición de reflexión de cadenas, ahora consideremos la cadena $ua$, por definición de reflexión se tiene que:\\

\quad $\bullet (ua)^R=au^R$\\

\textbf{Punto 2:} Para generalizar la propiedad consideramos la cadena:\\

$$(u_1u_2...u_n)$$

luego:

\begin{align*}
(u_1u_2...u_n)^R&=(u_{n-1}u_n)^R(u_1u_2...u_{n-2})^R\\
&=u_n^Ru_{n-1}^R(u_1u_2...u_{n-2})^R
\end{align*}


Podemos repetir este proceso 2 a 2 paso a paso y llegamos a que:

$$(u_1u_2...u_n)^R=u_n^Ru_{n-1}^R...u_2^Ru_1^R$$\\

Uno podría pensar algo como $(u_1u_2...u_n)^R=u_nu_{n-1}...u_2u_1$, pero esta generalización falla ya que recordemos que los $u_i$ son cadenas y no elementos del lenguaje, es decir pueden representar una cadena de varios elementos y se hace necesario reflejarlos también, por ejemplo:\\

Sea $\Sigma=\{a,b,c\}$ y considere $u_1=ab$ y $u_2=ac$, $v=u_1u_2=abac$, entonces:

$$v^R=caba$$

Mientras que $(u_1u_2)^R=acab$ si tomamos esa definición, note que $v^R\neq (u_1u_2)^R$, por tanto se hace evidente porqué esa definición no sirve.

$\hfill \blacklozenge$



\chapter{Sección 1.7.}
%!TEX root = main.tex
\textbf{Punto 1:} $u=d b d$\\

Como en el alfabeto solo hay a,b,c,d y $a<b<c<d$, entonces la cadena que buscamos es $v=dca$ ya que e no existe en el alfabeto y el siguiente a $b$ es $c$ y cambiamos la $d$ por a ya que necesitamos la cadena que sigue y $dca<dcb<dcc<dcd$.\\

Similarmente analizamos los siguientes ejercicios:\\

\textbf{Punto 2:} $u=acbd$ aplicando el análisis anterior encontramos $v=acca$\\

\textbf{Punto 3:} $u=dabcdd$, nuevamente encontramos $v=dabdaa$\\

\textbf{Punto 4:} $dcaddd$, finalmente $v=dcbaaa$

$\hfill \blacklozenge$

\chapter{Sección 1.10.}
%!TEX root = main.tex

\textbf{Punto 1:} Considere $\displaystyle\Sigma=\{a,b\}$ los lenguajes $A=$\O \text{ }y $B={a}$, por definición de concatenación de lenguajes:

$$AB=\text{\O}=BA$$

\textbf{Punto 2:} Sea $\displaystyle\Sigma=\{a,b,c\}$, considere los lenguajes $A=\{a\}$, $B=\{b\}$ y $C=\{c\}$, entonces:

\begin{align*}
    A\cdot(B\cap C)&=\text{\O}  &&\longleftarrow(A\cdot \text{\O = \O})\\
    &=\{ab\}\cap\{bc\}\\
\end{align*}
La intersección es vacía porque $ab$ como cadena es diferente de $bc$\\

\textbf{Punto 3:}

\begin{proof}
Sea $u\in A\cdot(B\cap C)$, entonces $u=a_ix_i$ para algún $a_i\in A$ y $x_i \in B\cap C$, luego como $x_i\in B\cap C$ $x_i\in B$ y $x_i \in C$, por tanto $u \in A\cdot B$ y $u\in A\cdot C$, así $u\in A\cdot B \cap A\cdot C$, concluimos que $A \cdot(B \cap C) \subseteq A \cdot B \cap A \cdot C$

\end{proof}


$\hfill \blacklozenge$
\chapter{Sección 1.12.}
\textbf{Punto 1:} Esto ocurre ya que la clausura de Kleene son todas las posibles concatenaciones de elementos de un lenguaje, por tanto unir las de dos lenguajes contempla las clausuras de ambos por separado y luego las une, mientras que unir los lenguajes y luego hacer la clausura contempla todas las posibles concatenaciones de elementos que se encontraban en ambos lenguajes, de hecho se podría ver con un argumento similar que $(A\cup B)^*\supset A^* \cup B^*$.\\


\textbf{Punto 2:} En este caso ocurre algo similar y es que faltan cadenas de $(A \cup B)^*$, por ejemplo considere $A=\{a\}$ y $B=\{b\}$, entonces por ejemplo es imposible obtener la cadena $(ba)^2$ a traves de $A^* \cup B^* \cup A^* B^* \cup B^* A^*$, estos mismos lenguajes sirven para darnos cuenta en el punto 1 que $(A \cup B)^*\supset A^*\cup B^*$, es este caso es lo mismo, $(A \cup B)^* \supset$ $A^* \cup B^* \cup A^* B^* \cup B^* A^*$.\\

Para ver la falsedad de estas afirmaciones recomendamos siempre usar lenguajes pequeños (con pocas cadenas) de tal manera que no nos gastemos mucho tiempo haciendo cuentas. 


$\hfill \blacklozenge$
\chapter{Sección 1.13.}
Los ejercicios de esta sección son más bien opcionales ya que Korgi no suele evaluar demostraciones en este curso, sin embargo para las personas que quizá se les dificulte demostrar estas afirmaciones y esté interesado en aprender se hace la solución de los mismos.\\

\textbf{Punto 1:}

\begin{itemize}
    \item[2)]
    \begin{proof}
        Usando la definición de unión y de reflexión de un lenguaje tenemos que:

\begin{align*}
    (A \cup B)^R&=\{u^R: u \in A \text{ o } u\in B\}\\
    &=\{x^R: x\in A\}\cup\{y^R: y \in B\}\\
    &=A^R\cup B^R
\end{align*}  
    \end{proof}

    \item[3)] 

    \begin{proof}
        En este caso es similar el argumento solo que cambiamos la ''o'' por una ''y'' ya que es una intersección y por tanto los elementos deben estar en ambos conjuntos:

        \begin{align*}
              (A \cap B)^R&=\{u^R: u \in A \text{ y } u\in B\}\\
    &=\{x^R: x\in A\}\cap\{y^R: y \in B\}\\
    &=A^R\cap B^R
        \end{align*}
    \end{proof}

    \item[4)]\begin{proof}
        Usando la definición de reflexión de un lenguaje tenemos que:

        $$(A^R)^R=\{(u^R)^R:u\in A\}$$

        Es decir es la reflexión de la reflexión de todas las cadenas de A, y nosotros ya sabemos que la reflexión de la reflexión de una cadena es la misma cadena luego:

        $$(A^R)^R=\{u: u \in A\}=A$$

    \end{proof}

     \item[6)] 
   \begin{proof}En este caso seguiremos el mismo modelo de prueba que se usa en las notas de clase para la propiedad 5
        $$
    \begin{aligned}
    x \in\left(A^+\right)^R & \Longleftrightarrow x=u^R, \text { donde } u \in A^+ \\
    & \Longleftrightarrow x=\left(u_1 \cdot u_2 \cdots u_n\right)^R, \text { donde los } u_i \in A, n \geq 1 \\
    & \Longleftrightarrow x=u_n^R \cdot u_2^R \cdots u_1^R, \text { donde } \operatorname{los} u_i \in A, n \geq 1 \\
    & \Longleftrightarrow x \in\left(A^R\right)^+ .
    \end{aligned}
    $$
   \end{proof}
\end{itemize}

\textbf{Punto 2:} CLARAmente se pueden generalizar las propiedades 2 y 3 ya que la unión y la intersección se comportan bien 2 a 2, es decir si tenemos que calcular la reflexión de la unión de $n$ conjuntos, podemos hacerlo con los primeros dos y luego con los siguientes dos y así hasta acabar o que nos quede solo uno y pues en ese caso calculamos la reflexión y unimos todo o en el otro caso intersectamos.\\

\begin{align*}
\displaystyle\left(\bigcup_{i \geq 0} A_i\right)^R&=(A_1\cup A_2)^R\cup (A_3\cup A_4\cup A_5\cup ...)^R\\
&=A_1^R\cup A_2^R\cup (A_3\cup A_4)^R\cup (A_5\cup A_6\cup A_7\cup ...)^R
\end{align*}

Y continuando así llegamos a:

$$\displaystyle\left(\bigcup_{i \geq 0} A_i\right)^R=A_1^R\cup A_2^R\cup A_3^R\cup...\cup A_n^R...$$

De manera análoga se ve la intersección.

\hfill $\blacklozenge$
\chapter{Sección 2.2.}
En este capítulo comenzamos la segunda parte de las notas de clase, en estas secciones resaltamos que no hay una única solución a los ejercicios y que algunas de las soluciones pueden llegar a ser redundantes, sin embargo son funcionales y esto es lo que más nos interesa.\\


\textbf{Punto 1:} \begin{itemize}
    \item [✎] La solución más evidente es la siguiente: $b(a\cup b)^*a$, con $(a\cup b)^*$ consideramos todas las cadenas y lo que hacemos es forzar que las cadenas comiencen con $b$ y terminen en $a$ concatenando.

    \item[✎] Sabemos que para generar todas las cadenas de longitud par usamos $(aa\cup bb\cup ab\cup ba)^*=((a\cup b)(a\cup b))^*$, luego para generar las impares debemos considerar 4 casos y unirlos:

    $$a((a\cup b)(a\cup b))^*\cup b((a\cup b)(a\cup b))^* \cup ((a\cup b)(a\cup b))^*a \cup ((a\cup b)(a\cup b))^*b$$

    Esto convierte las cadenas pares en impares siempre y considera los casos en que comience por $a$ o por $b$ o termine por $a$ o $b$ (se puede llegar a una solución mejor quizá).

    \item[✎] Sabemos que $(b^*ab^*ab^*)^*$ genera todas las cadenas con un número par (mayor que 0) de $aes$, luego usando este hecho construimos:

    $$a(b^*ab^*ab^*)^*\cup (b^*ab^*ab^*)^*a$$

    y acabamos.

    \item[✎] Sabemos generar las cadenas de $a$ y $b$ que contienen un número par de $a$ o $b$, la solución está propuesta en las notas de clase, luego cambiamos un poco la expresión de esta forma:

    $$a^*(a^*ba^*ba^*ba^*)^*$$

    La expresión $a^*(ba^*ba^*b)^*a^*$ también es una solución.

    \item[✎] Para este caso las cadenas pueden comenzar por $a$, $ab$, $b^2$, $a^2$, luego obtenemos la expresión:

    $$(a\cup ab \cup b^2 \cup a^2)(a \cup b)^*a $$

    Ya que tampoco pueden acabar en $b$, ahora no falta notas que por agregar esta $a$ al final y la expresión $(a\cup ab \cup b^2 \cup a^2)$ es imposible obtener la cadena vacía y la cadena $a$, pues las añadimos y nos queda finalmente:

     $$(a\cup ab \cup b^2 \cup a^2)(a \cup b)^*a \cup a \cup \lambda$$

     \item[✎] Para la expresión regular en este caso notemos que toda cadena tiene bloques de la forma $ba$ donde estos van intercalados con un numero de $a$ arbitrarias así obtenemos la expresión:

     $$(a\cup ba)^*$$

     Observe que las $b$ están restringidas ya que para $b\geq 2$ las cadenas de este estilo no pueden tener una $a$ luego de la cantidad arbitraria de $bes$, pero esta expresión no contempla las cadenas del estilo $bb\dots b$, para esto basta concatenar estas cadenas al final, obteniendo finalmente la expresión:

     $$(a\cup ba)^*b^*$$

\end{itemize}

\textbf{Punto 2:} \begin{itemize}
    \item[✎] Al igual que en el primer ítem del punto anterior, lo mas natural es forzar que la cadena empiece en con $2$ y termine en $1$, concatenando respectivamente obtenemos la expresión $2(0\cup1\cup2)^*1$.

    \item[✎] Similar a la construcción anterior podemos forzar a que las cadenas no empiecen con $2$ ni terminen con $1$ concatenando $(0\cup1)$ y $(0\cup2)$ respectivamente. De esta forma obtenemos la expresión:
    
    $$(0\cup1)(0\cup1\cup2)^*(0\cup2)$$

    Note que en el lenguaje propuesto las cadenas $\lambda$ y $0$ también cumplen la condición, pero es imposible generarlas por medio de la expresión dada. Afortunadamente arreglar esto es sencillo ya que podemos agregarlas por medio de uniones, obteniendo así:

    $$(0\cup1)(0\cup1\cup2)^*(0\cup2)\cup\lambda\cup0$$

    \item[✎] Nuevamente la forma mas natural de construir la expresión que represente al lenguaje es forzando que aparezcan solo $2$ ceros, tenga en cuenta que los ceros pueden estar en cualquier posición y por tanto la expresión es la siguiente $(1\cup2)^*0(1\cup2)^*0(1\cup2)^*$.

    \item[✎] Ya sabemos como generar los bloques de dos elementos de un lenguaje, para este caso $(0\cup1\cup2)(0\cup1\cup2)=(0\cup1\cup2)^2$, luego de esto basta concatenar estos bloques de todas la formas posibles, obteniendo así la expresión:

    $$\left((0\cup1\cup2)^2\right)^*$$

    \item[✎] Usando la expresión del ítem anterior, si concatenamos al final $0$,$1$ o $2$ obtenemos las cadenas de longitud impar, es decir:
    
    $$\left((0\cup1\cup2)^2\right)^*(0\cup1\cup2)$$

    \item[✎] Como no pueden aparecer dos unos consecutivos, las cadenas contienen bloques de la forma $(0\cup2)1(0\cup2)$, y junto a ellas cantidades arbitrarias de ceros y dos alternados:

    $$(0\cup2\cup(0\cup2)1(0\cup2))^*$$

    Observe que si bien esta expresión nos da múltiples cadenas aun hay varias que no genera. Por ejemplo no genera cadenas que empiecen o terminen en $1$. Esto podemos agregarlo concatenado $1\cup\lambda$ al inicio y final de la expresión:

    $$(1\cup\lambda)(0\cup2\cup(0\cup2)1(0\cup2))^+(1\cup\lambda)$$

    La cadena $\lambda$ es de vital importancia en la expresión ya que esta nos permite concatenar sin perder las cadenas que ya teníamos previamente. Además note que en la expresión cambiamos la $*$ por un $+$, esto se debe a que si no realizamos este cambio generaríamos la cadena $11$ y esta no cumple los criterios del lenguaje, por ultimo las cadenas $\lambda$ y $1$ cumplen las condiciones, mas no pueden ser generadas por lo que solo queda agregarlas y así obtener la expresión final:

    $$(1\cup\lambda)(0\cup2\cup(0\cup2)1(0\cup2))^+(1\cup\lambda)\cup\lambda\cup1$$
\end{itemize}

\textbf{Punto 3:} \begin{itemize}
    \item[✎] Para que una cadena tenga al menos un $0$ y un $1$ debe ser mínimo un bloque $01$ o un bloque $10$, luego simplemente fijamos esas dos posibilidades para que la solución sea  $(0\cup1)^*(01\cup10)(0\cup1)^*$.

    \item[✎]La condición nos indica que en las cadenas solo pueden haber uno o dos ceros consecutivos, es decir las forman bloque de la forma $01$ o $001$, luego podemos generar la expresión:

    $$(1\cup01\cup001)^*$$

    Note que esta  no contempla cadenas que terminen en uno o dos ceros, pero esto lo podemos arreglar fácilmente concatenando lo necesario:

    $$(1\cup01\cup001)^*(\lambda\cup0\cup00)$$

    \item[✎]Para esta solución simplemente consideremos las cadenas de longitud $4$ es decir las que son generadas por $(0\cup1)^4$, luego como estas son las mínimas cadenas que acepta el lenguaje solo queda hacer que aparezcan las demás posibilidades así $(0\cup1)^4(0\cup1)^*$.

    \item[✎]Note que al forzar que el quinto símbolo de izquierda a derecha sea un $1$ en todas las cadenas básicamente podemos rehusar la solución anterior forzando la condición de esta forma $(0\cup1)^41(0\cup1)^*$.

    \item[✎]Si la cadena no puede terminar en $01$ forzosamente tiene que terminar en $00,10$ o $11$, forzando estas obtenemos:

    $$(0\cup1)^*(00\cup10\cup11)$$

    Observe que esta expresión solo genera cadenas de longitud $\geq2$ pero las cadenas $\lambda,0$ y $1$ cumplen la condición, entonces:

    $$(0\cup1)^*(00\cup10\cup11)\cup\lambda\cup0\cup1$$

    De esta forma terminamos.

    \item[✎]Como son cadenas de longitud par, pueden ser formadas por bloques de la forma $01$ o de la forma $10$ luego la solución luce de esta forma $(01)^+\cup(10)^+$. Note que usamos $+$ ya que la cadena $\lambda$ no es aceptada en este lenguaje.

    \item[✎]Como en el ejercicio anterior ya construimos las cadenas pares, solo nos queda construir todas las impares. Esto lo logramos por medio de concatenar un elemento mas a las expresiones que ya tenemos. La solución luce de la siguiente forma:

    $$(1\cup\lambda)(01)^+\cup(0\cup\lambda)(10)^+$$

    \item[✎]Note que si no pueden haber dos ceros seguidos ni dos unos seguidos, los ceros y los unos deben de ir alternados forzosamente, es decir que la solución es la misma que la del ejercicio previo, exceptuando un detalle:

     $$(1\cup\lambda)(01)^*\cup(0\cup\lambda)(10)^*$$

     Observe que cambiamos el $+$ por una $*$, esto se debe a que las cadenas $\lambda,0$ y $1$ si son validas en este lenguaje.

     \item[✎]Para generar cadenas cuya longitud es un múltiplo de 3, necesitamos todos los bloques de longitud $3$ y posteriormente los concatenamos de todas las formas posibles, es decir tenemos la expresión $\left((0\cup1)^3\right)^*$. Recuerde que $0$ es múltiplo de $3$ por eso usamos el $*$ para asegurar la cadena $\lambda$.

      \item[✎]Esta expresión sigue un análisis muy similar al de las cadenas donde no podían haber tres ceros consecutivos, de esta forma solo falta agregar los bloques $0001$ y $000$ respectivamente a la expresión que habíamos obtenido:

      $$(1\cup01\cup001\cup0001)^*(\lambda\cup0\cup00\cup000)$$

      \item[✎]Observe que la cadena tiene que empezar por $1$ o $01$ y de forma similar tiene que acabar en $0$ o en $01$. Forzando estos símbolos obtenemos:

      $$(1\cup01)(0\cup1)^*(0\cup01)$$

      Ahora como usualmente ha ocurrido a lo largo de esta sección, al forzar cadenas en la expresión, no generamos cadenas que si son aceptadas dentro del lenguaje, pero basta con simplemente agregarlas:

      $$(1\cup01)(0\cup1)^*(0\cup01)\cup\lambda\cup0\cup1\cup01$$

      \item[✎]Para que no contengan la subcadena $101$ note que se debe forzar que en todas las expresiones aparezcan al menos dos ceros entre dos unos, las cadenas de este estilo se consiguen por medio de la expresión:

      $$(1\cup00^+)^*$$

      Uno podría verse tentado en pensar que esta es la solución, pero observe que esta expresión no contempla cadenas que empiecen por $01$ y que son totalmente validas, además tampoco contempla cadenas que terminen en un solo cero:

      $$(01\cup\lambda)(1\cup00^+)^*(0\cup\lambda)$$

      Luego de este arreglo si podemos asegurar que están todas las cadenas.
  
\end{itemize}

\hfill $\blacklozenge$
\chapter{Sección 2.5.}
Al igual que en la sección anterior las soluciones de este capitulo no son únicas y puede que algunas sean redundantes, además otra aclaración de vital importancia es que todos los autómatas presentados no se muestran sus estados limbo, es decir presentaremos AFD simplificados.\\

\textbf{Punto 1:}
\begin{itemize}
    \item[✎] Como la cadena mínima es $\lambda$ entonces el estado inicial tiene que ser de aceptación, luego como también se aceptan $aes$ arbitrarias estas pueden ser aceptadas por medio de un bucle. Apenas aparezca una $b$ el autómata cambiara de estado pero ese seria también de aceptación, incluyendo un bucle para las $bes$ arbitrarias:\\
    \begin{center}
        \begin{tikzpicture}[node distance = 2.5cm, on grid, auto]
            \node (q0) [state, initial, accepting] {$q_0$};
            \node (q1) [state, right of=q0, accepting] {$q_1$};
            \path [thick]
            (q0) edge [loop above] node {$a$} ()
            (q0) edge node [above] {$b$} (q1)
            (q1) edge [loop above] node {$b$} ();
        \end{tikzpicture}
    \end{center}

    \item[✎] Nuevamente el estado inicial es de aceptación ya que $\lambda$ pertenece a el lenguaje, luego basta con tomar dos caminos para el caso donde sean cadenas de $aes$ y el de cadenas de $bes$:
    \begin{center}
        \begin{tikzpicture}[node distance = 2cm, on grid, auto]
            \node (q0) [state, initial, accepting] {$q_0$};
            \node (q1) [state, above right of=q0, accepting] {$q_1$};
            \node (q2) [state, below right of=q0, accepting] {$q_2$};
            \path[thick]
            (q0) edge [bend left] node [above left] {$a$} (q1)
            (q0) edge [bend right] node [below left] {$b$} (q2)
            (q1) edge [loop above] node {$a$} ()
            (q2) edge [loop below] node {$b$} ();
        \end{tikzpicture}    
    \end{center}

    \item[✎] Note que todas las cadenas de este lenguaje son de la forma $ab\dots ab$, es decir siempre son bloques $ab$ y todas las cadenas empiezan en $a$ y terminan en $b$:
    \begin{center}
        \begin{tikzpicture}[node distance = 2.5cm, on grid, auto]
            \node (q0) [state, initial, accepting] {$q_0$};
            \node (q1) [state, right of=q0] {$q_1$};
            \path[thick]
            (q0) edge [bend left] node [above] {$a$} (q1)
            (q1) edge [bend left] node [below] {$b$} (q0);
        \end{tikzpicture}
    \end{center}

    \item[✎] Bastante similar a la anterior excepto que la cadena mínima aceptada es $ab$ debido al $+$, así que forzamos esa cadena:
    \begin{center}
        \begin{tikzpicture} [node distance = 2.5cm, on grid, auto]
            \node (q0) [state, initial] {$q_0$};
            \node (q1) [state, right of=q0] {$q_1$};
            \node (q2) [state, accepting, right of=q1] {$q_2$};
            \path[thick]
            (q0) edge node [above] {$a$} (q1)
            (q1) edge [bend left] node [above] {$b$} (q2)
            (q2) edge [bend left] node [below] {$a$} (q1);
        \end{tikzpicture}     
    \end{center}

    \item[✎]Note que debido a la expresión se forman dos caminos, uno son las cadenas que empiezan por $a$ y luego tienen una cantidad de $bes$ arbitrarias. El otro son aquellas que comienzan por un numero de $bes$ arbitrarias pero están forzadas a terminar en $a$ para ser aceptadas:
    \begin{center}
        \begin{tikzpicture} [node distance = 2.5cm, on grid, auto]
            \node (q0) [state, initial] {$q_0$};
            \node (q1) [state, below right of=q0, accepting] {$q_1$};
            \node (q2) [state, above right of=q0] {$q_2$};
            \node (q3) [state,  below right of=q2, accepting] {$q_3$};
            \path[thick]
            (q0) edge [bend left] node [above left] {$b$} (q2)
            (q0) edge [bend right] node [below left] {$a$} (q1)
            (q1) edge [loop below] node [below] {$b$} ()
            (q2) edge [loop above] node [above] {$b$} ()
            (q2) edge [bend left] node [above right] {$a$} (q3);
        \end{tikzpicture}
    \end{center}

    \item[✎] test
    
    
\end{itemize}



\end{document}
