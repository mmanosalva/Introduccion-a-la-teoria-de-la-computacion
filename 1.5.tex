Este manual de solución de las notas de introducción a la teoría de la computación tiene como objetivo ser de ayuda para estudiantes que se encuentren viendo la materia en futuros semestres, a la fecha (2023-1) no hay ningún manual de solución de los ejercicios de este curso y nos pareció importante poder dar un poco de ayuda o soporte a las personas que en el futuro vean el curso, todo ha sido separado en secciones para que sea más sencillo encontrar los ejercicios específicos que el estudiante pueda necesitar.\\

Las soluciones que presentemos no necesariamente son correctas pero trabajaremos para que lo sean y ante cualquier error nos pueden escribir a los correos que están en la portada.\\


\textbf{Punto 1:} Primero consideramos la cadena vacía, es claro que \\

\quad $\bullet$ $\lambda^R=\lambda$\\

Esto por definición de reflexión de cadenas, ahora consideremos la cadena $ua$, por definición de reflexión se tiene que:\\

\quad $\bullet (ua)^R=au^R$\\

\textbf{Punto 2:} Para generalizar la propiedad consideramos la cadena:\\

$$(u_1u_2...u_n)$$

luego:

\begin{align*}
(u_1u_2...u_n)^R&=(u_{n-1}u_n)^R(u_1u_2...u_{n-2})^R\\
&=u_n^Ru_{n-1}^R(u_1u_2...u_{n-2})^R
\end{align*}


Podemos repetir este proceso 2 a 2 paso a paso y llegamos a que:

$$(u_1u_2...u_n)^R=u_n^Ru_{n-1}^R...u_2^Ru_1^R$$\\

Uno podría pensar algo como $(u_1u_2...u_n)^R=u_nu_{n-1}...u_2u_1$, pero esta generalización falla ya que recordemos que los $u_i$ son cadenas y no elementos del lenguaje, es decir pueden representar una cadena de varios elementos y se hace necesario reflejarlos también, por ejemplo:\\

Sea $\Sigma=\{a,b,c\}$ y considere $u_1=ab$ y $u_2=ac$, $v=u_1u_2=abac$, entonces:

$$v^R=caba$$

Mientras que $(u_1u_2)^R=acab$ si tomamos esa definición, note que $v^R\neq (u_1u_2)^R$, por tanto se hace evidente porqué esa definición no sirve.

$\hfill \blacklozenge$


