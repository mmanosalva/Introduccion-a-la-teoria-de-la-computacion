%!TEX root = main.tex
En este capítulo comenzamos la segunda parte de las notas de clase, en estas secciones resaltamos que no hay una única solución a los ejercicios y que algunas de las soluciones pueden llegar a ser redundantes, sin embargo son funcionales y esto es lo que más nos interesa.\\


\textbf{Punto 1:} \begin{itemize}
    \item [✎] La solución más evidente es la siguiente: $b(a\cup b)^*a$, con $(a\cup b)^*$ consideramos todas las cadenas y lo que hacemos es forzar que las cadenas comiencen con $b$ y terminen en $a$ concatenando.

    \item[✎] Sabemos que para generar todas las cadenas de longitud par usamos $(aa\cup bb\cup ab\cup ba)^*=((a\cup b)(a\cup b))^*$, luego para generar las impares debemos considerar 4 casos y unirlos:

    $$a((a\cup b)(a\cup b))^*\cup b((a\cup b)(a\cup b))^* \cup ((a\cup b)(a\cup b))^*a \cup ((a\cup b)(a\cup b))^*b$$

    Esto convierte las cadenas pares en impares siempre y considera los casos en que comience por $a$ o por $b$ o termine por $a$ o $b$ (se puede llegar a una solución mejor quizá).

    \item[✎] Sabemos que $(b^*ab^*ab^*)^*$ genera todas las cadenas con un número par (mayor que 0) de $aes$, luego usando este hecho construimos:

    $$a(b^*ab^*ab^*)^*\cup (b^*ab^*ab^*)^*a$$

    y acabamos.

    \item[✎] Sabemos generar las cadenas de $a$ y $b$ que contienen un número par de $a$ o $b$, la solución está propuesta en las notas de clase, luego cambiamos un poco la expresión de esta forma:

    $$a^*(a^*ba^*ba^*ba^*)^*$$

    La expresión $a^*(ba^*ba^*b)^*a^*$ también es una solución.

    \item[✎] Para este caso las cadenas pueden comenzar por $a$, $ab$, $b^2$, $a^2$, luego obtenemos la expresión:

    $$(a\cup ab \cup b^2 \cup a^2)(a \cup b)^*a $$

    Ya que tampoco pueden acabar en $b$, ahora no falta notas que por agregar esta $a$ al final y la expresión $(a\cup ab \cup b^2 \cup a^2)$ es imposible obtener la cadena vacía y la cadena $a$, pues las añadimos y nos queda finalmente:

     $$(a\cup ab \cup b^2 \cup a^2)(a \cup b)^*a \cup a \cup \lambda$$

     \item[✎] Para la expresión regular en este caso notemos que toda cadena tiene bloques de la forma $ba$ donde estos van intercalados con un numero de $a$ arbitrarias así obtenemos la expresión:

     $$(a\cup ba)^*$$

     Observe que las $b$ están restringidas ya que para $b\geq 2$ las cadenas de este estilo no pueden tener una $a$ luego de la cantidad arbitraria de $bes$, pero esta expresión no contempla las cadenas del estilo $bb\dots b$, para esto basta concatenar estas cadenas al final, obteniendo finalmente la expresión:

     $$(a\cup ba)^*b^*$$

\end{itemize}

\textbf{Punto 2:} \begin{itemize}
    \item[✎] Al igual que en el primer ítem del punto anterior, lo mas natural es forzar que la cadena empiece en con $2$ y termine en $1$, concatenando respectivamente obtenemos la expresión $2(0\cup1\cup2)^*1$.

    \item[✎] Similar a la construcción anterior podemos forzar a que las cadenas no empiecen con $2$ ni terminen con $1$ concatenando $(0\cup1)$ y $(0\cup2)$ respectivamente. De esta forma obtenemos la expresión:
    
    $$(0\cup1)(0\cup1\cup2)^*(0\cup2)$$

    Note que en el lenguaje propuesto las cadenas $\lambda$ y $0$ también cumplen la condición, pero es imposible generarlas por medio de la expresión dada. Afortunadamente arreglar esto es sencillo ya que podemos agregarlas por medio de uniones, obteniendo así:

    $$(0\cup1)(0\cup1\cup2)^*(0\cup2)\cup\lambda\cup0$$

    \item[✎] Nuevamente la forma mas natural de construir la expresión que represente al lenguaje es forzando que aparezcan solo $2$ ceros, tenga en cuenta que los ceros pueden estar en cualquier posición y por tanto la expresión es la siguiente $(1\cup2)^*0(1\cup2)^*0(1\cup2)^*$.

    \item[✎] Ya sabemos como generar los bloques de dos elementos de un lenguaje, para este caso $(0\cup1\cup2)(0\cup1\cup2)=(0\cup1\cup2)^2$, luego de esto basta concatenar estos bloques de todas la formas posibles, obteniendo así la expresión:

    $$\left((0\cup1\cup2)^2\right)^*$$

    \item[✎] Usando la expresión del ítem anterior, si concatenamos al final $0$,$1$ o $2$ obtenemos las cadenas de longitud impar, es decir:
    
    $$\left((0\cup1\cup2)^2\right)^*(0\cup1\cup2)$$

    \item[✎] Como no pueden aparecer dos unos consecutivos, las cadenas contienen bloques de la forma $(0\cup2)1(0\cup2)$, y junto a ellas cantidades arbitrarias de ceros y dos alternados:

    $$(0\cup2\cup(0\cup2)1(0\cup2))^*$$

    Observe que si bien esta expresión nos da múltiples cadenas aun hay varias que no genera. Por ejemplo no genera cadenas que empiecen o terminen en $1$. Esto podemos agregarlo concatenado $1\cup\lambda$ al inicio y final de la expresión:

    $$(1\cup\lambda)(0\cup2\cup(0\cup2)1(0\cup2))^+(1\cup\lambda)$$

    La cadena $\lambda$ es de vital importancia en la expresión ya que esta nos permite concatenar sin perder las cadenas que ya teníamos previamente. Además note que en la expresión cambiamos la $*$ por un $+$, esto se debe a que si no realizamos este cambio generaríamos la cadena $11$ y esta no cumple los criterios del lenguaje, por ultimo las cadenas $\lambda$ y $1$ cumplen las condiciones, mas no pueden ser generadas por lo que solo queda agregarlas y así obtener la expresión final:

    $$(1\cup\lambda)(0\cup2\cup(0\cup2)1(0\cup2))^+(1\cup\lambda)\cup\lambda\cup1$$
\end{itemize}

\textbf{Punto 3:} \begin{itemize}
    \item[✎] Para que una cadena tenga al menos un $0$ y un $1$ debe ser mínimo un bloque $01$ o un bloque $10$, luego simplemente fijamos esas dos posibilidades para que la solución sea  $(0\cup1)^*(01\cup10)(0\cup1)^*$.

    \item[✎]La condición nos indica que en las cadenas solo pueden haber uno o dos ceros consecutivos, es decir las forman bloque de la forma $01$ o $001$, luego podemos generar la expresión:

    $$(1\cup01\cup001)^*$$

    Note que esta  no contempla cadenas que terminen en uno o dos ceros, pero esto lo podemos arreglar fácilmente concatenando lo necesario:

    $$(1\cup01\cup001)^*(\lambda\cup0\cup00)$$

    \item[✎]Para esta solución simplemente consideremos las cadenas de longitud $4$ es decir las que son generadas por $(0\cup1)^4$, luego como estas son las mínimas cadenas que acepta el lenguaje solo queda hacer que aparezcan las demás posibilidades así $(0\cup1)^4(0\cup1)^*$.

    \item[✎]Note que al forzar que el quinto símbolo de izquierda a derecha sea un $1$ en todas las cadenas básicamente podemos rehusar la solución anterior forzando la condición de esta forma $(0\cup1)^41(0\cup1)^*$.

    \item[✎]Si la cadena no puede terminar en $01$ forzosamente tiene que terminar en $00,10$ o $11$, forzando estas obtenemos:

    $$(0\cup1)^*(00\cup10\cup11)$$

    Observe que esta expresión solo genera cadenas de longitud $\geq2$ pero las cadenas $\lambda,0$ y $1$ cumplen la condición, entonces:

    $$(0\cup1)^*(00\cup10\cup11)\cup\lambda\cup0\cup1$$

    De esta forma terminamos.

    \item[✎]Como son cadenas de longitud par, pueden ser formadas por bloques de la forma $01$ o de la forma $10$ luego la solución luce de esta forma $(01)^+\cup(10)^+$. Note que usamos $+$ ya que la cadena $\lambda$ no es aceptada en este lenguaje.

    \item[✎]Como en el ejercicio anterior ya construimos las cadenas pares, solo nos queda construir todas las impares. Esto lo logramos por medio de concatenar un elemento mas a las expresiones que ya tenemos. La solución luce de la siguiente forma:

    $$(1\cup\lambda)(01)^+\cup(0\cup\lambda)(10)^+$$

    \item[✎]Note que si no pueden haber dos ceros seguidos ni dos unos seguidos, los ceros y los unos deben de ir alternados forzosamente, es decir que la solución es la misma que la del ejercicio previo, exceptuando un detalle:

     $$(1\cup\lambda)(01)^*\cup(0\cup\lambda)(10)^*$$

     Observe que cambiamos el $+$ por una $*$, esto se debe a que las cadenas $\lambda,0$ y $1$ si son validas en este lenguaje.

     \item[✎]Para generar cadenas cuya longitud es un múltiplo de 3, necesitamos todos los bloques de longitud $3$ y posteriormente los concatenamos de todas las formas posibles, es decir tenemos la expresión $\left((0\cup1)^3\right)^*$. Recuerde que $0$ es múltiplo de $3$ por eso usamos el $*$ para asegurar la cadena $\lambda$.

      \item[✎]Esta expresión sigue un análisis muy similar al de las cadenas donde no podían haber tres ceros consecutivos, de esta forma solo falta agregar los bloques $0001$ y $000$ respectivamente a la expresión que habíamos obtenido:

      $$(1\cup01\cup001\cup0001)^*(\lambda\cup0\cup00\cup000)$$

      \item[✎]Observe que la cadena tiene que empezar por $1$ o $01$ y de forma similar tiene que acabar en $0$ o en $01$. Forzando estos símbolos obtenemos:

      $$(1\cup01)(0\cup1)^*(0\cup01)$$

      Ahora como usualmente ha ocurrido a lo largo de esta sección, al forzar cadenas en la expresión, no generamos cadenas que si son aceptadas dentro del lenguaje, pero basta con simplemente agregarlas:

      $$(1\cup01)(0\cup1)^*(0\cup01)\cup\lambda\cup0\cup1\cup01$$

      \item[✎]Para que no contengan la subcadena $101$ note que se debe forzar que en todas las expresiones aparezcan al menos dos ceros entre dos unos, las cadenas de este estilo se consiguen por medio de la expresión:

      $$(1\cup00^+)^*$$

      Uno podría verse tentado en pensar que esta es la solución, pero observe que esta expresión no contempla cadenas que empiecen por $01$ y que son totalmente validas, además tampoco contempla cadenas que terminen en un solo cero:

      $$(01\cup\lambda)(1\cup00^+)^*(0\cup\lambda)$$

      Luego de este arreglo si podemos asegurar que están todas las cadenas.
  
\end{itemize}

\hfill $\blacklozenge$