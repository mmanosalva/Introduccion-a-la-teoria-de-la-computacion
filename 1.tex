%!TEX root = main.tex
Este manual de solución de las notas de introducción a la teoría de la computación tiene como objetivo ser de ayuda para estudiantes que se encuentren viendo la materia en futuros semestres, a la fecha (2023-1) no hay ningún manual de solución de los ejercicios de este curso y nos pareció importante poder dar un poco de ayuda o soporte a las personas que en el futuro vean el curso, todo ha sido separado en secciones para que sea más sencillo encontrar los ejercicios específicos que el estudiante pueda necesitar.\\

Las soluciones que presentemos no necesariamente son correctas pero trabajaremos para que lo sean y ante cualquier error nos pueden escribir a los correos que están en la portada.

\section{Sección 1.5.}

\textbf{Punto 1:} Primero consideramos la cadena vacía, es claro que \\

\quad $\bullet$ $\lambda^R=\lambda$\\

Esto por definición de reflexión de cadenas, ahora consideremos la cadena $ua$, por definición de reflexión se tiene que:\\

\quad $\bullet (ua)^R=au^R$\\

\textbf{Punto 2:} Para generalizar la propiedad consideramos la cadena:\\

$$(u_1u_2...u_n)$$

luego:

\begin{align*}
(u_1u_2...u_n)^R&=(u_{n-1}u_n)^R(u_1u_2...u_{n-2})^R\\
&=u_n^Ru_{n-1}^R(u_1u_2...u_{n-2})^R
\end{align*}


Podemos repetir este proceso 2 a 2 paso a paso y llegamos a que:

$$(u_1u_2...u_n)^R=u_n^Ru_{n-1}^R...u_2^Ru_1^R$$\\

Uno podría pensar algo como $(u_1u_2...u_n)^R=u_nu_{n-1}...u_2u_1$, pero esta generalización falla ya que recordemos que los $u_i$ son cadenas y no elementos del lenguaje, es decir pueden representar una cadena de varios elementos y se hace necesario reflejarlos también, por ejemplo:\\

Sea $\Sigma=\{a,b,c\}$ y considere $u_1=ab$ y $u_2=ac$, $v=u_1u_2=abac$, entonces:

$$v^R=caba$$

Mientras que $(u_1u_2)^R=acab$ si tomamos esa definición, note que $v^R\neq (u_1u_2)^R$, por tanto se hace evidente porqué esa definición no sirve.

$\hfill \blacklozenge$

\section{Sección 1.7}

\textbf{Punto 1:} $u=d b d$\\

Como en el alfabeto solo hay a,b,c,d y $a<b<c<d$, entonces la cadena que buscamos es $v=dca$ ya que e no existe en el alfabeto y el siguiente a $b$ es $c$ y cambiamos la $d$ por a ya que necesitamos la cadena que sigue y $dca<dcb<dcc<dcd$.\\

Similarmente analizamos los siguientes ejercicios:\\

\textbf{Punto 2:} $u=acbd$ aplicando el análisis anterior encontramos $v=acca$\\

\textbf{Punto 3:} $u=dabcdd$, nuevamente encontramos $v=dabdaa$\\

\textbf{Punto 4:} $dcaddd$, finalmente $v=dcbaaa$

$\hfill \blacklozenge$

\section{Sección 1.10} 

\textbf{Punto 1:} Considere $\displaystyle\Sigma=\{a,b\}$ los lenguajes $A=$\O \text{ }y $B={a}$, por definición de concatenación de lenguajes:

$$AB=\text{\O}=BA$$

\textbf{Punto 2:} Sea $\displaystyle\Sigma=\{a,b,c\}$, considere los lenguajes $A=\{a\}$, $B=\{b\}$ y $C=\{c\}$, entonces:

\begin{align*}
    A\cdot(B\cap C)&=\text{\O}  &&\longleftarrow(A\cdot \text{\O = \O})\\
    &=\{ab\}\cap\{bc\}\\
\end{align*}
La intersección es vacía porque $ab$ como cadena es diferente de $bc$\\

\textbf{Punto 3:}

\begin{proof}
Sea $u\in A\cdot(B\cap C)$, entonces $u=a_ix_i$ para algún $a_i\in A$ y $x_i \in B\cap C$, luego como $x_i\in B\cap C$ $x_i\in B$ y $x_i \in C$, por tanto $u \in A\cdot B$ y $u\in A\cdot C$, así $u\in A\cdot B \cap A\cdot C$, concluimos que $A \cdot(B \cap C) \subseteq A \cdot B \cap A \cdot C$

\end{proof}


$\hfill \blacklozenge$

\section{Sección 1.12.}

\textbf{Punto 1:} Esto ocurre ya que la clausura de Kleene son todas las posibles concatenaciones de elementos de un lenguaje, por tanto unir las de dos lenguajes contempla las clausuras de ambos por separado y luego las une, mientras que unir los lenguajes y luego hacer la clausura contempla todas las posibles concatenaciones de elementos que se encontraban en ambos lenguajes, de hecho se podría ver con un argumento similar que $(A\cup B)^*\supset A^* \cup B^*$.\\


\textbf{Punto 2:} En este caso ocurre algo similar y es que faltan cadenas de $(A \cup B)^*$, por ejemplo considere $A=\{a\}$ y $B=\{b\}$, entonces por ejemplo es imposible obtener la cadena $(ba)^2$ a traves de $A^* \cup B^* \cup A^* B^* \cup B^* A^*$, estos mismos lenguajes sirven para darnos cuenta en el punto 1 que $(A \cup B)^*\supset A^*\cup B^*$, es este caso es lo mismo, $(A \cup B)^* \supset$ $A^* \cup B^* \cup A^* B^* \cup B^* A^*$.\\

Para ver la falsedad de estas afirmaciones recomendamos siempre usar lenguajes pequeños (con pocas cadenas) de tal manera que no nos gastemos mucho tiempo haciendo cuentas. 


$\hfill \blacklozenge$

\section{Sección 1.13.}

Los ejercicios de esta sección son más bien opcionales ya que Korgi no suele evaluar demostraciones en este curso, sin embargo para las personas que quizá se les dificulte demostrar estas afirmaciones y esté interesado en aprender se hace la solución de los mismos.\\

\textbf{Punto 1:}

\begin{itemize}
    \item[2)]
    \begin{proof}
        Usando la definición de unión y de reflexión de un lenguaje tenemos que:

\begin{align*}
    (A \cup B)^R&=\{u^R: u \in A \text{ o } u\in B\}\\
    &=\{x^R: x\in A\}\cup\{y^R: y \in B\}\\
    &=A^R\cup B^R
\end{align*}  
    \end{proof}

    \item[3)] 

    \begin{proof}
        En este caso es similar el argumento solo que cambiamos la ''o'' por una ''y'' ya que es una intersección y por tanto los elementos deben estar en ambos conjuntos:

        \begin{align*}
              (A \cap B)^R&=\{u^R: u \in A \text{ y } u\in B\}\\
    &=\{x^R: x\in A\}\cap\{y^R: y \in B\}\\
    &=A^R\cap B^R
        \end{align*}
    \end{proof}

    \item[4)]\begin{proof}
        Usando la definición de reflexión de un lenguaje tenemos que:

        $$(A^R)^R=\{(u^R)^R:u\in A\}$$

        Es decir es la reflexión de la reflexión de todas las cadenas de A, y nosotros ya sabemos que la reflexión de la reflexión de una cadena es la misma cadena luego:

        $$(A^R)^R=\{u: u \in A\}=A$$

    \end{proof}

     \item[6)] 
   \begin{proof}En este caso seguiremos el mismo modelo de prueba que se usa en las notas de clase para la propiedad 5
        $$
    \begin{aligned}
    x \in\left(A^+\right)^R & \Longleftrightarrow x=u^R, \text { donde } u \in A^+ \\
    & \Longleftrightarrow x=\left(u_1 \cdot u_2 \cdots u_n\right)^R, \text { donde los } u_i \in A, n \geq 1 \\
    & \Longleftrightarrow x=u_n^R \cdot u_2^R \cdots u_1^R, \text { donde } \operatorname{los} u_i \in A, n \geq 1 \\
    & \Longleftrightarrow x \in\left(A^R\right)^+ .
    \end{aligned}
    $$
   \end{proof}
\end{itemize}

\textbf{Punto 2:} CLARAmente se pueden generalizar las propiedades 2 y 3 ya que la unión y la intersección se comportan bien 2 a 2, es decir si tenemos que calcular la reflexión de la unión de $n$ conjuntos, podemos hacerlo con los primeros dos y luego con los siguientes dos y así hasta acabar o que nos quede solo uno y pues en ese caso calculamos la reflexión y unimos todo o en el otro caso intersectamos.\\

\begin{align*}
\displaystyle\left(\bigcup_{i \geq 0} A_i\right)^R&=(A_1\cup A_2)^R\cup (A_3\cup A_4\cup A_5\cup ...)^R\\
&=A_1^R\cup A_2^R\cup (A_3\cup A_4)^R\cup (A_5\cup A_6\cup A_7\cup ...)^R
\end{align*}

Y continuando así llegamos a:

$$\displaystyle\left(\bigcup_{i \geq 0} A_i\right)^R=A_1^R\cup A_2^R\cup A_3^R\cup...\cup A_n^R...$$

De manera análoga se ve la intersección.

\hfill $\blacklozenge$


